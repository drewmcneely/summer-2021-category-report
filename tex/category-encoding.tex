\section{Encoding Markov Categories as an Interface}
\subsection{Preliminaries: Markov Categories}
As outlined by Fritz in \cite{fritz}, a Markov category is defined to be a symmetric monoidal category in which each object has a comonoidal structure.
An extra axiom that will be beneficial for our purposes is the existence of conditionals.
We will encode this as an operation on morphisms.

To unpack this definition a little further, we should go into detail on the implications of each definition.
First, a Markov category is a category.
Mathematically, this is simply a class of objects with morphisms between them that form a directed graph upon which there is a certain algebra of paths.
The important parts of the algebra are that each object must contain an identity morphism, and morphisms can compose.
Concretely, objects can be used to represent structured spaces while morphisms are structure preserving transformations.
A concrete category is a collection of such spaces with their transformations, typically given a name to represent the structure it encodes.

\subsection{Preliminaries: Python Constructs}
\subsubsection{Infix Operators}
\subsubsection{Abstract Base Classes}

\subsection{Class Hierarchy}
\subsubsection{Category class}
In computer science applications, the objects in a category often represent data types, and morphisms are the methods, routines, or functions that take a certain data type as input and return another as output.
This paradigm alone does not have enough structure for our needs.
To represent the structure of a category, it is sufficient to encode a representation of the morphisms only and to leave out the objects.
We will do this here: we will create a datatype representing morphisms, ie.\ stochastic kernels, and a datatype to use as an index for the objects (ie.\ spaces of probability distributions), but we will \emph{not} create a representation of probability distributions themselves.
This will be expounded upon in 

As such, we can make a datatype for a category that has a constructor for morphisms, and need only to define methods that allow them to compose and that give us an identity for each object.

\begin{lstlisting}[caption=The category class within category.py, label=category-class, float, floatplacement=H, language=Python, frame=single]
class Category(ABC):
    @abstractmethod
    def __init__(self, *args, **kwargs): pass

    @abstractmethod
    def compose(self, other): pass
    def __matmul__(self,other): return self.compose(other)

    @classmethod
    @abstractmethod
    def identity(cls, obj): pass

    @property
    @abstractmethod
    def source(self): pass
    @property
    @abstractmethod
    def target(self): pass
\end{lstlisting}

Listing \ref{category-class} shows how the Category class is implemented.
We have abstract methods specifying that for any data type inheriting from \lstinline{Category}, methods must be implemented for composition and identity.
\lstinline{compose} is a normal method, so the programmer should define it such that \lstinline{G.compose(F)} will return their composition.
The class also provides a wrapper for Python's matrix multiplication operator.
The programmer will subsequently be able to call \lstinline{G @ F} and get the same result.
The identity method is a class method that takes in an object $X$ and returns $\mathrm{id}_X$.
For the if the programmer defines a class called \lstinline{MyCat} that inherits from \lstinline{Category}, then identity would be called as \lstinline{MyCat.identity(X)}.

The remaining two property methods should be implemented to specify the domain and codomain of a kernel.

\subsubsection{Monoidal class}

Markov categories are monoidal categories, meaning that their objects act as a monoid.
In particular, for every 
\subsubsection{Symmetric class}
\subsubsection{Markov class}
