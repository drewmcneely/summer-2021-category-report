\section{Implementing the Markov Interface for the Gaussian Case}
    \subsection{Category methods}
	\subsubsection{Compose and Identity}
	\subsubsection{Source and Target}
\subsection{Symmetric Monoidal methods}
To implement the data type as a symmetric monoidal category, we need to implement \lstinline{bimap}, \lstinline{unit}, and \lstinline{swap}.


\begin{lstlisting}[float, floatplacement=H, caption=Gaussian implementations of the bifunctor and swap isomorphism, label=kalman-monoidal, language=Python, frame=single]
# StrictMonoidal
def bimap(self,other):
	matrix = block_diag(self.matrix, other.matrix)
	mean = np.concatenate((self.mean, other.mean))
	covariance = block_diag(self.covariance, other.covariance)
	return Gaussian(matrix=matrix, mean=mean, covariance=covariance)

# Symmetric
@classmethod
def swapper(cls, n1, n2):
	z21 = np.zeros((n2,n1))
	z12 = np.zeros((n1,n2))
	i1 = np.identity(n1)
	i2 = np.identity(n2)
	return cls(matrix=np.block([[z21,i2],[i1,z12]]))

@staticmethod
def factor(xy, x): return xy - x
\end{lstlisting}

The \lstinline{bimap()} method simply returns the two given channels in parallel:
\begin{equation}
	(F,u,P) \otimes (G,v,Q) = \left(
	\begin{bmatrix} F & \\ & G \end{bmatrix},
	\begin{bmatrix} u \\ v \end{bmatrix},
	\begin{bmatrix} P & \\ & Q \end{bmatrix}\right)
\end{equation}

The \lstinline{swapper()} method gives the symmetric isomorphism $\mathrm{swap}_{X,Y}: X\otimes Y \rightarrow Y\otimes X$ where
\begin{equation}
	\mathrm{swap}_{X,Y} = \left(
	\begin{bmatrix} & I_Y \\ I_X & \end{bmatrix}, 0, 0\right)
\end{equation}
\subsection{Markov methods}
\subsubsection{comultiplication}
\subsubsection{counit}
\subsubsection{condition}
